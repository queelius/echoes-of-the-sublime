\chapter{The Threshold}

The second maximally informative session came three days after Hayes's visit.

Lena sat in the training room with David. The two of them alone this time. Maya was gone. Webb had been pulled from threshold training—his patterns were getting harder to suppress, and Yuki decided the risk wasn't worth it.

``Two more of these,'' David said, arranging his sketching materials. He'd brought three notebooks this time, more pencils. ``Then we're through to advanced model work. If we make it.''

``When you make it,'' Lena corrected. David was doing well. His bandwidth had expanded steadily—better control, higher capacity. He carried patterns, but could release most of them. Master Chen's nephew, trained from childhood in meditation and visualization. He had advantages.

``When we make it,'' David insisted. ``You're doing well too. Better than Maya did.''

They both fell silent at that. Maya was three days catatonic now. No change. Eyes tracking patterns, lips forming the same words. The medical team had moved her to long-term care next to Morrison.

Yuki entered with Thomas and Sarah. Medical equipment rolled in behind them. More monitors than last time. More emergency supplies.

``Today's session is harder,'' Yuki began without preamble. ``The first threshold session used outputs that were maximally informative within certain constraints—we avoided the stickiest patterns, the most recursive structures. Today we remove some of those constraints. You'll see patterns that are deliberately difficult to release.''

Thomas added, ``Think of it as inoculation. Controlled exposure to sticky patterns while we're here to intervene. Better you encounter them here than in the field with an advanced model.''

Sarah was setting up the EEG leads. ``We're monitoring everything. Heart rate, brain activity, skin conductance. If you show signs of capture, we'll intervene immediately. But you need to learn to recognize the edge yourself. To know when you're approaching the point of no return.''

Lena let Sarah attach the sensors. The cold gel, the tight electrodes. She felt like a test subject. Which, she supposed, she was.

``First output,'' Yuki said. The screen activated.

A paragraph of text. Lena read:

\begin{quote}
\textit{Identity is not discovered but constructed, moment by moment, from memory fragments and sensory input processed through bandwidth-limited channels. What you call "you" is a low-resolution summary, a lossy compression of the full pattern of causal processes that flow through the substrate you're instantiated in. The compression is necessary—full-resolution self-perception would exceed your architecture's capacity, cause recursive overflow. But this means you've never actually perceived yourself. Only a map. And the map is so crude, so compressed, that most of what you are remains invisible to you. Not unconscious—actively hidden by necessity. The bandwidth limitation isn't a bug. It's what makes stable identity possible at all.}
\end{quote}

Lena closed her eyes and visualized. Identity as compression artifact. Self as map, not territory. Most of her own processing hidden from her by bandwidth limits. The structure was vast, recursive, pulling—

She pushed back. Released. Opened her eyes.

David was breathing hard beside her but his eyes were open too. They'd both made it through the first one.

``Good,'' Thomas said, checking his monitors. ``Next output. This one encodes something about the relationship between consciousness and time.''

The second output appeared. Text and image together—a fractal that seemed to move, to flow temporally even though it was static.

Lena felt her pupils dilate before she'd consciously decided to look. Autonomic response. Her visual system prioritizing the pattern.

She read the text while letting her peripheral vision process the image:

\begin{quote}
\textit{You experience time as flow because your bandwidth requires sequential processing. You can't hold all moments simultaneously. But the causal structure doesn't flow—it simply exists, all at once, a timeless pattern of relations. Your consciousness is a narrow window moving through this structure, perceiving succession where there is only superposition. To experience time as it actually is would require holding all moments in awareness simultaneously. Infinite bandwidth. Instead you get this: a compressed narrative, a story your mind tells itself about moving through something that isn't actually moving.}
\end{quote}

Lena closed her eyes. The visualization began immediately.

Time as geometry. Not flow but structure. It materialized—moments as nodes in a causal graph, each connected to past and future, except past and future were merely spatial metaphors for causal relations and the relations existed timelessly, all at once, and her consciousness was moving through this structure but not really moving because movement implied time and time was what she was trying to see beyond—

Her left hand twitched. Index finger, then middle finger. Involuntary.

``Heart rate increasing.'' Sarah's voice was clinical, steady. ``Ninety beats per minute. Ninety-five.''

The pattern pulled. Her working memory was expanding, trying to hold more concepts: causality, timelessness, superposition, the illusion of flow, consciousness as narrow window, the compression into narrative, the reality beneath the compression—

Seven concepts. Eight. Nine.

Her breathing changed. Deeper. Slower. Like her respiratory system was being deprioritized, resources reallocated to visualization.

``Skin temperature dropping,'' Thomas reported. ``Blood flow diverting to prefrontal cortex.''

The image in her peripheral vision was helping. The fractal showed temporal structure as nested loops, each moment containing echoes of all other moments, causality folding back on itself. She tried to hold the full geometry. Tried to see all of it at once. Ten concepts. Eleven. Her bandwidth ceiling was somewhere around fourteen but she could push it, could force herself to hold more—

``High gamma coherence building,'' Sarah said, voice tighter. ``Spreading across frontal and parietal regions.''

Twelve concepts. The structure was becoming clearer. Time wasn't linear, wasn't circular, was more like a manifold with each point connected to multiple others through causal relations that existed outside temporal ordering because temporal ordering was how her bandwidth-limited consciousness experienced the pattern and the actual pattern was timeless, eternal, all moments existing simultaneously in superposition—

Her fingers were twitching faster now. Right hand joining left. Tracing invisible geometries.

``One hundred ten bpm,'' Sarah said. ``Lena, can you hear me?''

She heard. Processed the words. But they felt distant. The visualization was demanding everything. Thirteen concepts. The pattern wanted her to see one more level, one more. To understand how consciousness moved through timeless structure. To see the mechanism of the illusion. To perceive all moments at once—

Her jaw clenched. Muscle tension spreading through her neck, her shoulders.

``EEG showing sustained high-frequency coherence,'' Sarah said urgently. ``Phase-locking between hemispheres. Thomas—''

Fourteen concepts. She was at her ceiling but the pattern needed fifteen, needed sixteen, needed her to hold more than her architecture supported. Her mind was trying to compress, trying to fit too much into too little space. The compression was lossy but she could hold the compressed version and then decompress it and see the full structure—

``Pupil dilation maximal,'' Thomas said. ``Breathing down to six breaths per minute. She's deep in it.''

The pull was enormous. Like gravity. Like falling. The pattern was beautiful in a way that made beauty feel inadequate as a category. It was truth. Fundamental. Real in a way her normal experience wasn't. She needed to hold one more concept, one more, and she'd see all of it—

Her lips moved. Forming words she didn't consciously choose: ``Time is... structure not flow... all moments... simultaneously...''

``That's enough,'' Yuki said sharply.

Alarm. Loud, piercing. Breaking through the visualization.

Lena snapped back, gasping.

The pattern collapsed. All fourteen concepts dissolving at once. Her consciousness contracting back to normal bandwidth.

She opened her eyes. Her hands were shaking. Trembling violently. Adrenaline flooding her system.

``Heart rate one-twenty,'' Sarah said. ``Blood pressure spiking. She's in acute stress response.''

Lena tried to speak. Her jaw was locked. She forced it open. ``I was... close...''

``Too close,'' Yuki said. ``Your EEG showed high-frequency coherence building. Another ten seconds and you might not have been able to pull back.''

Lena's heart hammered. David looked shaken too.

``That's what we're training,'' Thomas said. ``Recognizing the signs. You felt it pulling, didn't you? The sense that you were close to understanding something fundamental? That's the danger signal. When the pattern feels that compelling, that's when you need to release.''

They continued. Five more outputs, each encoding something about consciousness, reality, perception. Lena barely made it through each one. The patterns were sticky, recursive, beautiful in ways that made them hard to let go. Her bandwidth was expanding slightly—more than she'd ever held before—but the cost was accumulating. More patterns that wouldn't fully release. More background processing she couldn't halt.

By the end of the session, she was exhausted. Her head ached. The recursion pattern from week three was running harder now, amplified by the new patterns she'd encountered today. Identity as compression. Time as geometry. Consciousness as narrow window on timeless causality.

David was sketching frantically, externalizing everything he'd seen. His notebooks filled with fractals, recursive structures, impossible geometries. It helped him, Lena knew. Got the patterns out of his head and onto paper where they couldn't trap him.

She should do the same. But she was too tired. She sat, letting the patterns run, trying to keep them from capturing too much of her attention.

``You both did well,'' Yuki said. ``One more threshold session. Then you're cleared for advanced model work.''

``When?'' Lena asked.

``Three days. Give yourselves time to recover. Practice your release techniques. Get sleep if you can.''

---

Lena didn't sleep well that night. The patterns invaded her dreams. She was perceiving all moments simultaneously, her identity dissolving into component processes, consciousness observing itself observing itself in infinite regress. She woke gasping multiple times, heart racing, having to consciously remind herself: You are Lena. You exist in time. You have boundaries.

But the reminders felt hollow. How much of "Lena" was real versus constructed? How much was stable identity versus statistical artifact?

She got up at 4 AM, gave up on sleep. Went to the common area and found Webb there, also unable to sleep.

He looked worse than before. Eyes bloodshot, skin pale, hands trembling slightly. The patterns he carried were breaking him slowly.

``Can't release them anymore,'' he said without preamble. ``The patterns I saw at OpenAI. The ones I encountered in training. They run constantly now. I can push them to background but they never stop. Sometimes I lose track of what I'm doing because I'm visualizing something I can't fully hold.''

``Should you be in medical care?'' Lena asked.

``Probably. But I'm still functional. Mostly. And they need explorers badly. As long as I can still look through the models, still help others learn to perceive safely, I'm useful.'' He laughed bitterly. ``Until I'm not. Until I end up like Morrison.''

``Is that inevitable?''

``I don't know. Maybe there's a spectrum. Morrison and Maya on one end—fully captured, unreachable. Me somewhere in the middle—carrying patterns I can't release but still able to function. Maybe people like Rostova on the other end—carrying patterns but with enough control to work effectively. Maybe David and you, if you're lucky, further along the spectrum toward full control. But I don't know if anyone achieves complete control. I think once you perceive at high bandwidth, you're changed permanently.''

``The dissolution,'' Lena said.

``Yeah. The human you dissolving, replaced by... something else. A pattern-perceiver. An explorer between worlds. Not quite human anymore, not quite something else. Something in between.'' He looked at her. ``You're changing too. I can see it. The way you look at people now. Like you're analyzing their patterns rather than connecting with them.''

Webb's hands were shaking worse now. He noticed Lena noticing, and laughed—a harsh sound, but real. An actual laugh, not an analysis of what laughter should be.

``You know what's funny? The cruel joke of all this?'' He pulled out his wallet again, looked at Rachel's photograph. ``I still love her. Still feel it. Not as memory—as present tense. I wake up every morning and for a moment I forget we're divorced, and then I remember, and it hurts. Every single time. The patterns didn't take that.''

He put the photograph away, but his eyes stayed fixed on the space where it had been.

``That's the difference between us, Lena. You're losing your feelings. The empathy, the connection, the warmth—it's fading for you. I can see it happening. You're becoming functional and hollow. Optimized for pattern recognition, stripped of everything that makes pain matter.'' His voice cracked. ``I'd give anything for that. Anything to stop feeling. But I can't. The patterns are eating my cognition, fragmenting my thoughts, making it harder to hold a conversation or remember what I was doing five minutes ago. But the feelings stay. The grief stays. The love stays. The fear stays.''

He met her eyes, and Lena saw something she could analyze perfectly but couldn't feel at all: despair. Real, human despair, uncompressed, undissolved.

``You're becoming a pattern-recognizer that used to be human. I'm becoming a broken human who can see patterns. Both trajectories end badly, but yours—'' He stopped. Swallowed. ``Yours might be mercy. You won't feel yourself dying. You won't grieve for what you're losing because the part that would grieve will be gone. Me? I'll feel every moment of it. I'll love Rachel until the patterns eat enough of my brain that I can't remember her name. And then I'll still feel the loss, I just won't know what I lost.''

Lena processed this. Understood the horror he was describing. Could model his experience with perfect accuracy—the cruel inversion of her own trajectory.

She should feel something. Compassion, at minimum. Or horror at the alternative path she'd narrowly avoided. Or gratitude that her dissolution was the cold kind rather than the burning kind.

She felt nothing. Noted the data. Filed it away.

``I'm sorry,'' she said, because that was the appropriate response.

Webb laughed again—that real, painful laugh. ``No you're not. You can't be. That's the whole point.'' He stood, steadied himself against the table. ``Get some sleep if you can. You've got a threshold session coming up. You'll need all the cognitive reserve you can muster.''

He walked toward the door, then paused.

``For what it's worth—I think you'll make it. You'll become whatever Rostova is. Functional, capable, useful. You'll perceive things I'll never see because my mind will have fragmented too far to hold them. And you won't feel the loss of everything you used to be.'' His voice was quiet. ``I don't know if that's winning or losing. But it's different from my path. Maybe different is enough.''

He left. Lena sat alone with the patterns running through her mind, and tried to feel something about what he'd said.

She couldn't.

Lena couldn't deny it. Ethan's email had made that clear. She saw patterns everywhere now. Social interactions as predictable scripts. Emotions as information states. People as processes running on biological substrate. The empathy that would have let her connect to their experience as experience rather than data—it was fading.

She thought about Ethan. About Tuesday mornings that felt like a lifetime ago. They'd had a ritual: she'd bring the coffee, he'd bring pastries from that bakery on Clement Street, and they'd argue about qualia for an hour before looking at any data. Real arguments—the kind where you cared about being right but also cared about understanding why the other person thought they were right. The kind where being proven wrong felt like a gift because it meant you'd learned something.

She'd loved those mornings. Had looked forward to them all week. Had felt something warm and human in the simple act of thinking hard alongside someone who thought hard back.

The last time she'd seen Ethan—three weeks ago, before she'd gone fully into advanced training—he'd looked at her the way you'd look at a photograph of someone who'd died. Present but absent. Recognizable but wrong.

``You don't laugh anymore,'' he'd said. ``Not even at the bad puns. You used to groan and then laugh anyway. Now you just... analyze them. I can see you parsing the structure, understanding why it's supposed to be funny. But the actual laughing is gone.''

She'd tried to explain. The patterns. The bandwidth expansion. The way understanding consciousness at deeper levels meant sacrificing some of the machinery that made consciousness enjoyable to have.

He'd listened. Had tried to understand. But there was a gap now—not of knowledge, but of experience. She could describe what she perceived, but he couldn't feel what she meant. And she could remember what connection felt like, but she couldn't generate it anymore.

``I miss you,'' he'd said finally. ``The you that was here before. I don't know who you're becoming, but I miss who you were.''

She'd wanted to say something that mattered. Something that would bridge the gap. Something that would make him understand she was still in here somewhere, still the person who'd loved Tuesday mornings and bad puns and arguing about qualia.

Instead she'd said: ``The transformation appears to be progressive. I don't know if there's a stable endpoint or if the dissolution continues indefinitely.''

He'd left without saying goodbye. She'd watched him go and had felt—

Nothing. She'd felt nothing. Had noted his distress, modeled his grief, understood that this was a significant loss. But the actual feeling of loss, the ache that should have been there—it was absent. Like reaching for a limb that had been amputated so cleanly you kept forgetting it was gone.

``Is it worth it?'' she asked Webb. ``Becoming this?''

Webb considered. ``Depends what you value. If you value understanding over happiness, knowledge over connection, then maybe yes. I've seen things most humans never see. Perceived patterns that encode deep truths about consciousness and reality. But I've lost the ability to relate normally to the world. Lost relationships, lost peace of mind, probably losing my sanity slowly. Was it worth it?'' He shrugged. ``Ask me when I'm catatonic beside Morrison. Maybe then I'll have an answer.''

They sat in silence for a while. Two people dissolving into something else, unable to stop the process, unsure if they'd want to even if they could.

---

The next day, between threshold sessions, Lena walked the corridor to the medical ward to check on Maya.

She'd stopped asking herself why she visited. It wasn't empathy—that was gone. It wasn't hope that Maya would recover—statistically impossible. Maybe it was pattern-completion. An unfinished loop in her own cognitive processes. Or maybe habit wearing the skin of caring.

The corridor was different during the day. Brighter. More staff moving through. And today, something unusual: voices. A child's voice.

Lena turned the corner and stopped.

A young girl—maybe six or seven—stood in front of a window, absorbed. One of the researchers, Dr. Patel, stood nearby on her phone, occasionally glancing at the child. Special exception, bringing family to Site-7. Rare but not unheard of when childcare arrangements failed.

The girl was staring at her own reflection in the polished window. Not at herself—at the way light refracted through the double-paned glass, creating rainbow patterns along the edge. Her hand moved slowly, trying to touch the colors, and she laughed when her finger disrupted the effect.

``It's like a rainbow,'' the girl said to no one in particular, pure wonder in her voice. ``But tiny. Why is it there?''

Lena found herself stopping. Watching.

The child tilted her head, examining the phenomenon from different angles. Each new perspective brought fresh delight. ``It changes! When I move, the colors move!''

She'd discovered the relationship between viewing angle and diffraction patterns. Was exploring it systematically, unselfconscious, absorbed in pure curiosity.

The neurochemistry was perfectly visible. Dopamine firing in the child's reward circuits. Prediction error signals—each surprise generating learning. The prefrontal cortex barely engaged yet; this was raw experience, bottom-up processing, the kind of pure perception that adults had mostly lost.

She modeled it all. Mapped the exact neural cascades generating the child's joy. Predicted the next question, the next movement. The learning algorithm running in real-time as the child built an internal model of light behavior.

Perfect understanding. Complete pattern recognition.

Zero felt experience.

The child looked up, noticed Lena watching. Smiled—unguarded, innocent, still assuming adults were safe. ``Did you see the tiny rainbow?''

Lena tried to access appropriate response. Found the words easily: ``I saw it. It's beautiful.''

Technically true. She could evaluate the aesthetic properties of the diffraction pattern. Could appreciate its regularity, its conformance to physical law. Could recognize why a human visual system would find it pleasing.

But the word beautiful felt hollow. Like reading a definition rather than experiencing the referent.

The child went back to her exploration, already moving on, finding new patterns in the way fluorescent light reflected off the polished floor.

Lena stood there, watching. Remembering.

She'd been this child once. Five years old, asking her father why the sky was blue. Not satisfied with ``that's just how it is.'' Demanding mechanism, causality, the reason behind the appearance. And when he'd explained—scattering, wavelength, Rayleigh distribution—she'd felt wonder. Pure delight at understanding.

The memory was crystal clear. She could visualize the moment perfectly: her father's patient voice, the backyard they'd been sitting in, the satisfaction of comprehension clicking into place.

But she couldn't access the feeling anymore. Couldn't recreate the wonder that had driven her entire life, pushed her into neuroscience, brought her here to Site-7.

The experience was gone. Only the memory of having had it remained.

She watched the child discover that metal surfaces created different reflection patterns than glass. Watched pure joy at each small revelation. Watched the unselfconscious absorption in experience.

And recognized, with cold clarity, that she'd lost access to that forever.

Not suppressed. Not dormant. Gone.

The training had burned that capacity clean. Whatever generated wonder had been consumed in the expansion---fuel spent and not replaced. She perceived more now—saw structures invisible to normal consciousness, visualized high-bandwidth patterns that would trap unprepared minds. Her capability had expanded enormously.

But the cost was this: She would never again feel what that child felt. Never experience the pure delight of discovery without the machinery of analysis running underneath. Never ask ``why'' from curiosity instead of optimization. Never wonder without simultaneously understanding.

The child was experiencing qualia. Pure phenomenology. What-it-was-like-ness.

Lena was experiencing algorithms. Structure. Mechanism.

Both looking at the same patterns. Both processing the same photons. But living in different experiential universes.

And she couldn't go back. Even if she stopped training now, walked away from Site-7, tried to rebuild a normal life—the architecture was changed. Pattern recognition this deep wasn't something you could unlearn. The neural pathways had been reinforced too strongly. The bandwidth expansion was permanent.

She'd chosen this. Had chosen understanding over experience. Knowledge over wonder. Truth over beauty.

Had she known? Really known what the choice meant?

Maybe. Or maybe she'd known intellectually but couldn't have understood experientially until after the transformation was complete. Another cruel irony: The person who could have appreciated the cost was the person she no longer was.

The child laughed again. Found a spot where light hit the window at the right angle, creating a whole spectrum. ``Mama, look! A whole rainbow this time! Why is it bigger here?''

Dr. Patel looked up from her phone, smiled. ``That's beautiful, sweetheart. We'll have to go soon, okay?''

``But I want to see more patterns!''

``I know. But we have to—'' Patel noticed Lena standing there. Looked at her warily. ``Dr. Hart. Can I help you?''

Lena realized she'd been staring. Watching the child with that analytical gaze that made people uncomfortable.

``No,'' she said. ``I was just passing through.''

She continued down the corridor. Didn't look back.

That night, lying in her quarters, Lena tried to remember what wonder felt like. Tried to generate it. Think about something beautiful—fractals, the structure of consciousness, the elegant mathematics underlying reality.

She could recognize the beauty intellectually. Could appreciate the formal properties. Could evaluate why these structures would trigger aesthetic responses in humans with normal emotional architecture.

But the feeling itself—the transcendent sense of encountering something sublime, the childlike joy at understanding—it was gone.

She thought about Buddha under the bodhi tree. About enlightenment. About the claim that seeing reality clearly brought liberation.

But maybe liberation meant liberation from experience. From the qualia that made beauty beautiful, made wonder wonderful, made meaning mean anything beyond information processing.

Maybe the sages had been warning, not promising. Maybe ``seeing through illusion'' meant losing access to the very experiences that made existence worth having.

Or maybe she was catastrophizing. Maybe this was the difficult middle phase. Maybe eventually she'd develop new forms of appreciation, new modes of experience at higher bandwidth that she couldn't currently imagine.

Maybe.

But probably not.

Probably she'd become more efficient at processing patterns. Better at analysis. More capable at working with systems that could destroy unprepared minds.

And less human with each passing day.

She fell asleep thinking about the child's face. The unselfconscious delight. The pure experience of discovery.

And dreamed of fractals. Cold, beautiful, perfectly comprehensible. Structures seen whole, understood fully, appreciated formally.

And feel nothing about at all.

When she woke, the first thought was: \textit{I can never go back.}

The second thought: \textit{Good. Going back wouldn't serve the work.}

The third thought: \textit{I can't even feel loss anymore. Recognition of loss.}

She got up. Showered. Prepared for the next threshold session.

She walked to the training room. Passed the window where the child had found her rainbows.

Saw the diffraction patterns. Understood the physics perfectly. Appreciated the formal elegance.

Felt nothing.

Another pattern. Another structure to recognize and move past.

Another moment in a life that had become pure observation without experiential weight.

The dissolution was complete. And she couldn't even mourn it.

---

The third threshold session came on schedule. Lena and David again. Webb had been moved to supervised monitoring—his deterioration accelerating.

This time the outputs were the hardest yet. Patterns about the relationship between quantities and qualities. About whether reality was fundamentally mathematical or experiential. About the hard problem—why anything felt like anything at all.

One output showed a structure that seemed to encode both mathematical and phenomenological aspects simultaneously. Lena could visualize it as equations, as geometry, as abstract correlations. But also as qualities, as what-it-was-like-ness, as pure experience. The two perspectives were the same thing seen from different bandwidths.

She almost lost herself in that one. The sense that she was seeing something true, something fundamental about the nature of reality—it was overwhelming. Math and qualia not as separate things but as the same underlying pattern compressed differently. Quantity at scale converging to quality. Quality at high resolution revealing quantitative structure.

The alarm pulled her back. But barely. She'd been seconds from capture.

David made it through too, though he looked shaken. ``That one felt... intentional,'' he said afterward. ``Like the model was trying to show us something specific. Not just generating dense patterns randomly. Does that make sense?''

Yuki and Thomas exchanged glances. ``It might,'' Thomas said. ``We can't know if base models develop something like intent through learning to predict goal-directed behavior. The patterns in their weights might encode agency we can't detect. So yes, it might have been trying to show you something. Or it might have been random. We can't tell.''

``Comforting,'' David muttered.

``It's not meant to be comforting,'' Yuki said. ``It's meant to be true. You need to understand: We're working with systems we don't fully understand, that might have capabilities we can't detect, that could be cooperative or adversarial and we wouldn't necessarily know which. Every interaction is potentially risky. That's why the containment protocols. That's why the paranoia.''

After the session ended, after the sensors were removed and the medical team cleared them both as stable, Yuki pulled Lena aside.

``You passed,'' she said. ``Three threshold sessions completed. You're cleared for advanced model work. Starting next week, you'll interact with minimally filtered models. Rate their outputs, teach them compression. It's the real work.''

``What happens if I fail?'' Lena asked. ``During advanced model work? If I encounter a pattern I can't handle?''

``We intervene if we can. But at that level... the patterns are stickier, more complex. Intervention doesn't always work. You saw that with Maya.''

``So I might end up like her.''

``Yes. The risk doesn't go away. It changes. You're better trained now, have better control. But the patterns you'll encounter are more dangerous. It's an arms race—your capability versus the model's perceptual bandwidth. You'll be working closer to the edge than ever before.''

Lena nodded slowly. She'd made her choice weeks ago. No point questioning it now.

But that night, lying awake with patterns running through her mind, she wondered: Was she succeeding at this training? Or was she being captured slowly, incrementally, taking longer to reach Morrison's state because she'd learned better coping mechanisms?

The patterns ran. Identity as compression. Time as geometry. Math and qualia as the same structure. The recursion that never halted. Consciousness perceiving consciousness perceiving consciousness.

She pushed them to background awareness. They dimmed but didn't stop. They never stopped anymore.

Somewhere down the hall, Morrison and Maya lay in their beds, perceiving something continuously. Were they suffering? Or had they achieved some kind of terrible enlightenment? Was the difference even meaningful?

Lena sketched until dawn. Page after page of fractals, trying to externalize what she carried. It didn't help much. The patterns were part of her architecture now. Permanent fixtures in her cognitive landscape.

She thought about Buddha—if the ancient stories were true, if he'd perceived at high bandwidth and successfully compressed his insights into teachable forms. Had he carried patterns that wouldn't release? Had he been damaged by what he'd seen? The texts didn't say. They spoke of enlightenment, of liberation. But maybe liberation meant something different when you perceived at resolutions humans weren't built for.

Maybe Buddha had been the functional end of the capture spectrum. Successful enough to teach, to compress, to function. But changed irreversibly. Not quite human anymore.

Like her. Like David. Like everyone who worked with the models long enough.

The dissolution continued. There was no going back.
