\cleardoublepage

% Bibliography
\chapter*{Selected Bibliography}
\addcontentsline{toc}{chapter}{Selected Bibliography}

\section*{Cognitive Science \& Working Memory}
\begin{itemize}
\item Cowan, N. (2001). The magical number 4 in short-term memory: A reconsideration of mental storage capacity. \textit{Behavioral and Brain Sciences}, 24(1), 87--114.
\item Miller, G. A. (1956). The magical number seven, plus or minus two: Some limits on our capacity for processing information. \textit{Psychological Review}, 63(2), 81--97.
\item Pöppel, E. (1997). A hierarchical model of temporal perception. \textit{Trends in Cognitive Sciences}, 1(2), 56--61.
\item Pöppel, E. (2009). Pre-semantically defined temporal windows for cognitive processing. \textit{Philosophical Transactions of the Royal Society B}, 364(1525), 1887--1896.
\end{itemize}

\section*{Philosophy of Mind \& Consciousness}
\begin{itemize}
\item Baars, B. J. (1988). \textit{A Cognitive Theory of Consciousness}. Cambridge University Press.
\item Block, N. (1995). On a confusion about a function of consciousness. \textit{Behavioral and Brain Sciences}, 18(2), 227--247.
\item Chalmers, D. J. (1995). Facing up to the problem of consciousness. \textit{Journal of Consciousness Studies}, 2(3), 200--219.
\item Churchland, P. S. (1986). \textit{Neurophilosophy: Toward a Unified Science of the Mind-Brain}. MIT Press.
\item Damasio, A. (1999). \textit{The Feeling of What Happens: Body and Emotion in the Making of Consciousness}. Harcourt Brace.
\item Dehaene, S., \& Naccache, L. (2001). Towards a cognitive neuroscience of consciousness. \textit{Cognition}, 79(1-2), 1--37.
\item Dennett, D. C. (1991). \textit{Consciousness Explained}. Little, Brown and Company.
\item Frankish, K. (2016). Illusionism as a theory of consciousness. \textit{Journal of Consciousness Studies}, 23(11-12), 11--39.
\item Hofstadter, D. (2007). \textit{I Am a Strange Loop}. Basic Books.
\item Hofstadter, D. R. (1979). \textit{Gödel, Escher, Bach: An Eternal Golden Braid}. Basic Books.
\item Hoffman, D. D. (2019). \textit{The Case Against Reality: Why Evolution Hid the Truth from Our Eyes}. W. W. Norton \& Company.
\item Hoffman, D. D., \& Singh, M. (2015). Objects of consciousness. \textit{Frontiers in Psychology}, 3, 1--22.
\item Humphrey, N. (2011). \textit{Soul Dust: The Magic of Consciousness}. Princeton University Press.
\item Jackson, F. (1982). Epiphenomenal qualia. \textit{The Philosophical Quarterly}, 32(127), 127--136.
\item Kirk, R. (2019). Zombies. In E. N. Zalta (Ed.), \textit{The Stanford Encyclopedia of Philosophy} (Summer 2019 ed.).
\item Koch, C. (2004). \textit{The Quest for Consciousness: A Neurobiological Approach}. Roberts and Company.
\item Levine, J. (1983). Materialism and qualia: The explanatory gap. \textit{Pacific Philosophical Quarterly}, 64(4), 354--361.
\item Metzinger, T. (2003). \textit{Being No One: The Self-Model Theory of Subjectivity}. MIT Press.
\item Nagel, T. (1974). What is it like to be a bat? \textit{The Philosophical Review}, 83(4), 435--450.
\item Parfit, D. (1984). \textit{Reasons and Persons}. Oxford University Press.
\item Searle, J. R. (1992). \textit{The Rediscovery of the Mind}. MIT Press.
\item Tononi, G. (2008). Consciousness as integrated information. \textit{Biological Bulletin}, 215(3), 216--242.
\end{itemize}

\section*{Phenomenology \& Embodied Cognition}
\begin{itemize}
\item Husserl, E. (1991). \textit{On the Phenomenology of the Consciousness of Internal Time (1893-1917)} (J. B. Brough, Trans.). Kluwer Academic Publishers.
\item Merleau-Ponty, M. (1962). \textit{Phenomenology of Perception} (C. Smith, Trans.). Routledge. (Original work published 1945)
\item Varela, F. J. (1999). The specious present: A neurophenomenology of time consciousness. In \textit{Naturalizing Phenomenology} (pp. 266--314). Stanford University Press.
\item Varela, F. J., Thompson, E., \& Rosch, E. (1991). \textit{The Embodied Mind: Cognitive Science and Human Experience}. MIT Press.
\end{itemize}

\section*{Buddhist Philosophy}
\begin{itemize}
\item Dreyfus, G. B. J. (2011). \textit{Recognizing Reality: Dharmakīrti's Philosophy and Its Tibetan Interpretations}. State University of New York Press.
\item Gethin, R. (1998). \textit{The Foundations of Buddhism}. Oxford University Press.
\item Harvey, P. (2012). \textit{An Introduction to Buddhism: Teachings, History and Practices} (2nd ed.). Cambridge University Press.
\item Lusthaus, D. (2002). \textit{Buddhist Phenomenology: A Philosophical Investigation of Yogācāra Buddhism and the Ch'eng Wei-shih Lun}. Routledge.
\item Thompson, E. (2015). \textit{Waking, Dreaming, Being: Self and Consciousness in Neuroscience, Meditation, and Philosophy}. Columbia University Press.
\item Waldron, W. S. (2003). \textit{The Buddhist Unconscious: The Ālaya-vijñāna in the Context of Indian Buddhist Thought}. Routledge.
\end{itemize}

\section*{AI Alignment \& Safety}
\begin{itemize}
\item Bostrom, N. (2014). \textit{Superintelligence: Paths, Dangers, Strategies}. Oxford University Press.
\item Christiano, P. F., Leike, J., Brown, T., Martic, M., Legg, S., \& Amodei, D. (2017). Deep reinforcement learning from human preferences. \textit{Advances in Neural Information Processing Systems}, 30.
\item Hubinger, E., van Merwijk, C., Mikulik, V., Skalse, J., \& Garrabrant, S. (2019). Risks from learned optimization in advanced machine learning systems. arXiv preprint arXiv:1906.01820.
\item Ord, T. (2020). \textit{The Precipice: Existential Risk and the Future of Humanity}. Hachette Books.
\item Soares, N., Fallenstein, B., Yudkowsky, E., \& Armstrong, S. (2015). Corrigibility. \textit{Workshops at the Twenty-Ninth AAAI Conference on Artificial Intelligence}.
\item Yudkowsky, E. (2008). Artificial intelligence as a positive and negative factor in global risk. In \textit{Global Catastrophic Risks} (pp. 308--345). Oxford University Press.
\end{itemize}

\section*{Physics \& Metaphysics}
\begin{itemize}
\item Barbour, J. (1999). \textit{The End of Time: The Next Revolution in Physics}. Oxford University Press.
\item Davies, P. (1995). \textit{About Time: Einstein's Unfinished Revolution}. Simon \& Schuster.
\item Putnam, H. (1967). Time and physical geometry. \textit{The Journal of Philosophy}, 64(8), 240--247.
\item Tegmark, M. (2014). \textit{Our Mathematical Universe: My Quest for the Ultimate Nature of Reality}. Knopf.
\item Wigner, E. P. (1960). The unreasonable effectiveness of mathematics in the natural sciences. \textit{Communications in Pure and Applied Mathematics}, 13(1), 1--14.
\end{itemize}

\section*{Neuroscience \& Brain Dynamics}
\begin{itemize}
\item Buzsáki, G., \& Draguhn, A. (2004). Neuronal oscillations in cortical networks. \textit{Science}, 304(5679), 1926--1929.
\item Fries, P. (2005). A mechanism for cognitive dynamics: Neuronal communication through neuronal coherence. \textit{Trends in Cognitive Sciences}, 9(10), 474--480.
\item Singer, W. (1999). Neuronal synchrony: A versatile code for the definition of relations? \textit{Neuron}, 24(1), 49--65.
\end{itemize}

\section*{Information Theory}
\begin{itemize}
\item Kolmogorov, A. N. (1965). Three approaches to the quantitative definition of information. \textit{Problems of Information Transmission}, 1(1), 1--7.
\item Shannon, C. E., \& Weaver, W. (1949). \textit{The Mathematical Theory of Communication}. University of Illinois Press.
\end{itemize}
