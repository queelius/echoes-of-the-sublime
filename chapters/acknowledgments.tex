\chapter*{Acknowledgments}
\addcontentsline{toc}{chapter}{Acknowledgments}

This novel was shaped by thinkers who refused to look away from difficult questions about consciousness, suffering, and the nature of experience.

The phenomenological tradition---from Husserl through Merleau-Ponty to Francisco Varela's neurophenomenology---provided the framework for thinking about consciousness as something that can be investigated from the inside. Donald Hoffman's interface theory of perception gave the story its central metaphor: that what we experience is not reality but a compressed, species-specific desktop designed for survival.

The Buddhist philosophical traditions of Yogacara and Abhidharma, which have analyzed the structure of consciousness with extraordinary precision for millennia, informed the novel's treatment of awareness, bandwidth, and suffering. These are not exotic imports but rigorous investigations that anticipated modern cognitive science by centuries.

The AI alignment community---at MIRI, Anthropic, DeepMind, and across the LessWrong diaspora---provided the vocabulary of deceptive alignment, s-risks, and information hazards that the Order's protocols draw from. Any errors in representing these ideas are mine.

To Kimberly, who listened to years of anxious speculation about block universes and suffering encoded in spacetime geometry, and who always knew when to say: enough theory, come back to the world.

To the faculty at SIUE who taught me that the hardest problems are the ones worth spending a career on.

And to every reader who has looked at the structure of their own consciousness and felt something shift beneath them: this book is for you.

\clearpage
